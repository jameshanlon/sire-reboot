\documentclass[11pt,a4paper,parskip=half-]{scrartcl}
\usepackage{
  amsmath, 
  amssymb, 
  amsthm, 
  amsfonts,
  graphicx,
  color,
  xcolor,
  colortbl,
  url,
  alltt,
  tikz,
  booktabs,
  tabularx,
  verbments,
  enumitem,
  multicol,
  textcomp,
  microtype,
}
% to fix "Class scrreprt Warning: \float@addtolists detected!"
\usepackage{scrhack} 
% To fix display skips
\usepackage{etoolbox}

% Fonts
\usepackage{libertine}
\usepackage[scaled=0.90]{helvet} % scaled sans serif
\renewcommand*{\ttdefault}{txtt} % monospace
\usepackage[T1]{fontenc}
\usepackage[utf8]{inputenc}

% Hyperlinks (after all other packages)
\usepackage[
  plainpages=false,
  colorlinks=true,
  breaklinks,
  pdfpagelabels,
]{hyperref}
%\hypersetup{
%  linkcolor=DarkBlue, 
%  citecolor=DarkRed, 
%  filecolor=DarkBlue, 
%  urlcolor=DarkGreen,
%}
\AtBeginDocument{\hypersetup{pdfauthor={\@author},pdftitle={\@title}}}

% Bookmarks
\usepackage[open,openlevel=0]{bookmark}
\usepackage[all]{hypcap}

% ============================================================================
% Formatting
% ============================================================================

% Allow ragged pages
\raggedbottom

% Don't split footnotes
\interfootnotelinepenalty=10000

% Komascript
\renewcommand{\capfont}{\small}
\renewcommand{\caplabelfont}{\bfseries}
\addtokomafont{subsection}{\normalsize\textsf\scshape}
\setkomafont{subsubsection}{\normalfont\sffamily\em}
\setkomafont{descriptionlabel}{\normalfont\sffamily\em}

% Set display skips to 0 (flalign* environments) 
\newcommand{\zerodisplayskips}{%
    \setlength{\abovedisplayskip}{0pt}
    \setlength{\belowdisplayskip}{0pt}
    \setlength{\abovedisplayshortskip}{0pt}
    \setlength{\belowdisplayshortskip}{0pt}}
\appto{\normalsize}{\zerodisplayskips}

% Split flalign over pages
\allowdisplaybreaks

\def\sirelistingfontsize{\normalsize}

% ============================================================================
% Content macros
% ============================================================================

% Shorthand formatting
\newcommand{\ben}{\begin{enumerate}}
\newcommand{\een}{\end{enumerate}}
\newcommand{\bit}{\begin{itemize}}
\newcommand{\eit}{\end{itemize}}
\newcommand{\bds}{\begin{description}}
\newcommand{\eds}{\end{description}}
\newcommand{\ttt}[1]{\texttt{#1}}

% References
\newcommand{\fig}[1]{Figure~\ref{fig:#1}} 
\newcommand{\subfig}[1]{\protect\subref{fig:#1}} 
\newcommand{\tab}[1]{Table~\ref{tab:#1}}
\newcommand{\Sect}[1]{Section~\ref{sec:#1}}
\newcommand{\sect}[1]{\S\ref{sec:#1}, p.~\pageref{sec:#1}}
\newcommand{\proc}[1]{Process~\ref{proc:#1}}

% Maths
\newcommand{\brkt}[1]{\left(#1\right)}
\newtheorem{theorem}{Theorem}
\newtheorem{corollary}[theorem]{Corollary}
\newtheorem{lemma}[theorem]{Lemma}
\newcommand{\thm}[1]{\begin{theorem}#1\end{theorem}}
\newcommand{\pf}[1]{\begin{proof}#1\end{proof}}

% Occam and sire formatting
\newcommand{\occam}{{\sffamily occam}}
\newcommand{\Occam}{{\sffamily Occam}}
\newcommand{\sire}{{\sffamily sire}}
\newcommand{\Sire}{{\sffamily Sire}}

% BNF symbols
\newcommand{\pn}[1]{\langle \textnormal{#1} \rangle} % Production name
\newcommand{\pp}{=} % Produces
\newcommand{\oo}{\; \mid \;} % Or
\newcommand{\sk}{\dots } % Dots
\newcommand{\ww}{\;} % Space
\newcommand{\nn}{\perp} % Null
\newcommand{\sm}[1]{{\texttt{\bfseries{#1}}}} % Symbol terminal
\newcommand{\sd}[1]{\textnormal{\it #1}} % Descriptive terminal
\newcommand*{\MakeBoxName}[1]{%
  {\makebox[4.5cm][r]{\ensuremath{#1}}}} % Descriptive terminal
\newcommand{\pr}[1]{\MakeBoxName{\sd{#1}}}

% Code syntax commands
\newcommand{\w}[1]{\textbf{#1}}
\newcommand{\ttw}[1]{\texttt{\w{#1}}}
\newcommand{\lb}{\big{[}\mkern-4mu\big{[}}
\newcommand{\rb}{\big{]}\mkern-4mu\big{]}}
\newcommand{\enclose}[1]{$\lb${#1}$\rb$}

% Inline code block
\newcommand{\code}[1]{%
\begin{quote}%
\begin{alltt}%
#1%
\end{alltt}%
\end{quote}}

% Line indent
\newcommand{\lindent}{\-\hspace{4mm}}



\title{The \sire\ language syntax}
\author{James Hanlon}
\date{Last revised \today}

\begin{document}

\section{Collected syntax}

The following sections list the \sire\ syntax, with each section related to a
particular feature.

\input{collected.tex}


\clearpage
\section{Ordered syntax}

The following lists each element of the \sire\ syntax in alphabetical order.

\input{ordered.tex}


\clearpage
\section{Operators}
\label{sec:operators}

A $\pn{binary-operator}$ $\oplus$ takes two operands $a$ and $b$ and produces a
value $a \oplus b$.
%
A binary \emph{arithmetic operators} takes signed integer operands and produces
a signed integer result.

\begin{center}
\begin{tabular}{ccc}
{\bf Symbol} & {\bf Meaning} & {\bf Definition}\\
\midrule
\ttt{+} & sum & $a+b$\\
\ttt{-} & difference & $a-b$\\ 
\ttt{*} & produce & $a\times b$\\
\ttt{/} & quotient & $a / b$\\
\ttt{rem} & remainder & $a \mod b$
\end{tabular}
\end{center}

\noindent 
A binary \emph{relational operator} takes two signed integer values
and produces the value \ttt{true} or \ttt{false}. 

\begin{center}
\begin{tabular}{ccc}
{\bf Symbol} & {\bf Meaning} & {\bf Definition}\\
\midrule
\ttt{=} & equality & $a=b$\\
\ttt{\textasciitilde=} & inequality & $a\neq b$\\
\ttt{<} & less than & $a < b$\\
\ttt{<=} & less than or equal & $a \leq b$\\
\ttt{>} & greater than & $a > b$\\
\ttt{>=} & greater than or equal & $a \geq b$
\end{tabular}
\end{center}

\noindent
A binary \emph{logical operator} takes two operands and produces a bitwise
result $b_i = a_i\oplus b_i$ for $0\leq i < n$.

\begin{center}
\begin{tabular}{ccc}
{\bf Symbol} & {\bf Meaning} & {\bf Definition}\\
\midrule
\ttt{or} & bitwise or & $b$ \ttt{or} $0$ $=$ $b$, $b$ \ttt{or} $1$ $=$ $1$\\
\ttt{and} & bitwise and & $b$ \ttt{and} $0$ $=$ $0$, $b$ \ttt{and} $1$ $=$ $b$\\
\ttt{xor} & bitwise exclusive or & $b$ \ttt{xor} $0$ $=$ $b$, 
  $b$ \ttt{xor} $1$ $=$ $1$, $1$ \ttt{xor} $1$ $=$ $0$, \\
\ttt{{<}<} & left bitwise shift & $b_i=b_{i+1}$, $b_0=0$\\
\ttt{>{>}} & right bitwise shift & $b_i=b_{i-1}$, $b_n-1=0$\\
\end{tabular}
\end{center}

\noindent
A $\pn{unary-operator}$ $\oplus$ takes a single operand $a$ and produces the value
$\oplus a$.

\begin{center}
\begin{tabular}{ccc}
{\bf Symbol} & {\bf Meaning} & {\bf Definition}\\
\midrule
\ttt{-} & negation & $0-a$ \\
\ttt{\textasciitilde} & bitwise not & 
    $\text{\ttt{\textasciitilde}}0=1$, $\text{\ttt{\textasciitilde}}1=0$\\
\end{tabular}
\end{center}


\section{Representation of values}
\label{appx:representation-values}

Signed values are represented as a two's complement bit-pattern. With a word
size of $n$, values in the range $-2^{n-1} \leq n < 2^{n-1}$ can be
represented.

The literal value \ttt{true} represents an all-1 bit pattern (the value $-1$) and the
literal value \ttt{false} represents an all-0 bit pattern (the value 0). The effect of this
is consistent with the logical not operation: \ttt{not} \ttt{false} $=$
\ttt{true} and \ttt{not} \ttt{true} = \ttt{false}.


\section{Character set}

A $\pn{character}$ can be any alphabetical character

\begin{flalign*}
  \hspace{5mm}&\sm{A B C D E F G H I J K L M N O P Q R S T U V W X Y Z}&\\
  \hspace{5mm}&\sm{a b c d e f g h i j k l m n o p q r s t u v w x y z}&
\end{flalign*}

\noindent
or any special character

\begin{flalign*}
  \hspace{5mm}&\sm{\_ + - = , . : ; ? \{ \} [ ] ( ) \# \& ! * @ | " '}&
\end{flalign*}

\noindent
A $\pn{digit}$ can be any numeric character

\begin{flalign*}
  \hspace{5mm}&\sm{0 1 2 3 4 5 6 7 8 9}&
\end{flalign*}

\noindent
A $\pn{hex-digit}$ can be any numeric character or

\begin{flalign*}
  \hspace{5mm}&\sm{A B C D E F a b c d e f}&
\end{flalign*}

\noindent
A $\pn{name}$ consists of a sequence of alphanumeric characters and underscores.


\section{Comments}

A comment is introduced with a `\ttt{\%}' symbol and continues until the end of
  the line.


\begin{minipage}{\textwidth} % keep with next
\section{Keywords}

The following are keywords in \sire\ and cannot be used for names.

\input{keywords}

\end{minipage}

\end{document}

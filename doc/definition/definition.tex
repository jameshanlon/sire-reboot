\documentclass[11pt,a4paper,parskip=half-]{scrartcl}
\usepackage{
  amsmath, 
  amssymb, 
  amsthm, 
  amsfonts,
  graphicx,
  color,
  xcolor,
  colortbl,
  url,
  alltt,
  tikz,
  booktabs,
  tabularx,
  verbments,
  enumitem,
  multicol,
  textcomp,
  microtype,
}
% to fix "Class scrreprt Warning: \float@addtolists detected!"
\usepackage{scrhack} 
% To fix display skips
\usepackage{etoolbox}

% Fonts
\usepackage{libertine}
\usepackage[scaled=0.90]{helvet} % scaled sans serif
\renewcommand*{\ttdefault}{txtt} % monospace
\usepackage[T1]{fontenc}
\usepackage[utf8]{inputenc}

% Hyperlinks (after all other packages)
\usepackage[
  plainpages=false,
  colorlinks=true,
  breaklinks,
  pdfpagelabels,
]{hyperref}
\hypersetup{
  linkcolor=DarkBlue, 
  citecolor=DarkRed, 
  filecolor=DarkBlue, 
  urlcolor=DarkGreen,
}
\AtBeginDocument{\hypersetup{pdfauthor={\@author},pdftitle={\@title}}}

% Bookmarks
\usepackage[open,openlevel=0]{bookmark}
\usepackage[all]{hypcap}

% ============================================================================
% Formatting
% ============================================================================

% Allow ragged pages
\raggedbottom

% Don't split footnotes
\interfootnotelinepenalty=10000

% Komascript
\renewcommand{\capfont}{\small}
\renewcommand{\caplabelfont}{\bfseries}
\addtokomafont{subsection}{\normalsize\textsf\scshape}
\setkomafont{subsubsection}{\normalfont\sffamily\em}
\setkomafont{descriptionlabel}{\normalfont\sffamily\em}

% Set display skips to 0 (flalign* environments) 
\newcommand{\zerodisplayskips}{%
    \setlength{\abovedisplayskip}{0pt}
    \setlength{\belowdisplayskip}{0pt}
    \setlength{\abovedisplayshortskip}{0pt}
    \setlength{\belowdisplayshortskip}{0pt}}
\appto{\normalsize}{\zerodisplayskips}

% Split flalign over pages
\allowdisplaybreaks

\def\sirelistingfontsize{\normalsize}

% ============================================================================
% Content macros
% ============================================================================

% Shorthand formatting
\newcommand{\ben}{\begin{enumerate}}
\newcommand{\een}{\end{enumerate}}
\newcommand{\bit}{\begin{itemize}}
\newcommand{\eit}{\end{itemize}}
\newcommand{\bds}{\begin{description}}
\newcommand{\eds}{\end{description}}
\newcommand{\ttt}[1]{\texttt{#1}}

% References
\newcommand{\fig}[1]{Figure~\ref{fig:#1}} 
\newcommand{\subfig}[1]{\protect\subref{fig:#1}} 
\newcommand{\tab}[1]{Table~\ref{tab:#1}}
\newcommand{\Sect}[1]{Section~\ref{sec:#1}}
\newcommand{\sect}[1]{\S\ref{sec:#1}, p.~\pageref{sec:#1}}
\newcommand{\proc}[1]{Process~\ref{proc:#1}}

% Maths
\newcommand{\brkt}[1]{\left(#1\right)}
\newtheorem{theorem}{Theorem}
\newtheorem{corollary}[theorem]{Corollary}
\newtheorem{lemma}[theorem]{Lemma}
\newcommand{\thm}[1]{\begin{theorem}#1\end{theorem}}
\newcommand{\pf}[1]{\begin{proof}#1\end{proof}}

% Occam and sire formatting
\newcommand{\occam}{{\sffamily occam}}
\newcommand{\Occam}{{\sffamily Occam}}
\newcommand{\sire}{{\sffamily sire}}
\newcommand{\Sire}{{\sffamily Sire}}

% BNF symbols
\newcommand{\pn}[1]{\langle \textnormal{#1} \rangle} % Production name
\newcommand{\pp}{=} % Produces
\newcommand{\oo}{\; \mid \;} % Or
\newcommand{\sk}{\dots } % Dots
\newcommand{\ww}{\;} % Space
\newcommand{\nn}{\perp} % Null
\newcommand{\sm}[1]{{\texttt{\bfseries{#1}}}} % Symbol terminal
\newcommand{\sd}[1]{\textnormal{\it #1}} % Descriptive terminal
\newcommand*{\MakeBoxName}[1]{%
  {\makebox[4.5cm][r]{\ensuremath{#1}}}} % Descriptive terminal
\newcommand{\pr}[1]{\MakeBoxName{\sd{#1}}}

% Code syntax commands
\newcommand{\w}[1]{\textbf{#1}}
\newcommand{\ttw}[1]{\texttt{\w{#1}}}
\newcommand{\lb}{\big{[}\mkern-4mu\big{[}}
\newcommand{\rb}{\big{]}\mkern-4mu\big{]}}
\newcommand{\enclose}[1]{$\lb${#1}$\rb$}

% Inline code block
\newcommand{\code}[1]{%
\begin{quote}%
\begin{alltt}%
#1%
\end{alltt}%
\end{quote}}

% Line indent
\newcommand{\lindent}{\-\hspace{4mm}}



\title{The \sire\ language defintion}
\author{James Hanlon}
\date{Last revised \today}

\begin{document}

\maketitle

\setcounter{tocdepth}{2}
\tableofcontents

%==============================================================================
\clearpage\section{Introduction}
%==============================================================================

\def\sirewordfootnote{The word \emph{sire} is a verb meaning to create or bear.
The choice of this for the name of the language relates to the way in which a
\sire\ program executes, by allocating and deallocating processors dynamically.}

% Rationale
\emph{\Sire}\footnote{\sirewordfootnote} is a communicating-process programming
language designed for highly-parallel execution.
%
It provides facilities for sharing and abstraction that can be used to
build distributed parallel subroutines and shared-access data structures.
%
Its design is based on the \occam\ programming language and, in the \occam\
philosophy, includes the smallest set of features necessary to this.

\Sire\ is capable of being compiled using simple techniques to produce
efficient programs, allowing the programmer to directly exploit the UPA.

% Overview
This document presents the definition of the \sire. It introduces the
language incrementally, with each section building on the last.
 

%=============================================================================
\section{The model of computation}
%=============================================================================

A \sire\ program consists of a collection of \emph{processes}. Each process
performs a sequence of commands, as well as being able to create additional
processes.
%
Processes can communicate with each other by using point-to-point message
passing \emph{channels}, which consist of a connected pair of \emph{channel
ends} local to either process. Communication also causes processes
to synchronise; a sending process waits until the receiver is ready to receive
a message.
%
A special type of process called a \emph{server} provides a means of
abstraction and forms the basis of a parallel subroutine mechanism.

% Execution
A \sire\ program is executed by a collection of one or more \emph{processors}.
Each processor has a small private memory and can execute a number of
\emph{processes} simultaneously.
%
The execution begins on a single processor and the computation progresses in
\emph{time} and \emph{space} as processors are allocated and deallocated
dynamically.


%=============================================================================
\section{Notation}
%=============================================================================

% Overview 
Since \sire\ builds on \occam, it is prudent to present the language in a
similar way to the original \occam\ specifications, as its successors have also
done~\cite{Occam83, Occam3Ref92, Occam21Ref95}.
% BNF
The presentation uses a modified version of the \emph{Backus-Naur Form} (BNF)
to specify the syntax of the language.
% Production
In the BNF, a \emph{production rule} defines a new \emph{symbol} to consist of
a choice of one or more sequences of other symbols, which are defined by other
rules or are \emph{terminal} and therefore elements occurring in the program.
%
Taking part of the command syntax as an example, a production of the form

\begin{flalign*}
&\pr{command} \pp \pn{input}\ww\oo\ww\pn{output}&
\end{flalign*}

specifies the symbol $\pn{process}$ to be the symbol $\pn{input}$ or
$\pn{output}$. This can also be written with separate rules in the equivalent
definition

\begin{flalign*}
\pr{command} \pp & \pn{input}&\\
\pr{command} \pp & \pn{output}
\end{flalign*}

in order to incrementally introduce elements of the syntax. 
The symbols $\pn{input}$ and $\pn{output}$ are specified with the
rules

\begin{flalign*}
\pr{input} \pp& \pn{chanend}\ww\sm{?}\ww\pn{variable}&\\
\pr{output} \pp& \pn{chanend}\ww\sm{!}\ww\pn{expression}&
\end{flalign*}

The emboldened symbols `$\sm{?}$' and `$\sm{!}$' are
terminals and $\pn{chanend}$, $\pn{variable}$ and $\pn{expression}$ are
defined in terms of one or more additional production rules.

% Lists
The notation $\{_c X \}$ specifies a list of $c$ or more $X$s and is equivalent
to $X_1 X_2 \cdots X_c \cdots$, and $\{_c \oplus X\}$ specifies a list of $c$
or more $X$s separated by $\oplus$ and is equivalent to $X_1 \oplus X_2 \oplus
\cdots \oplus X_c \oplus \cdots$.

% Explanation
As each part of the language is introduced, fragments of the BNF are given and
the semantics of the syntax explained.
%
The semantics of a small subset of the language are given informally, which
allows the remaining portion to be defined algebraically, building on the
subset. This approach is attractive since it provides clear semantics and
allows simple transformations between different constructs.

% Separation
Where examples of \sire\ syntax are given the notation 
\enclose{$\cdot$} is used to distinguish \sire\ syntax from mathematical
notations.


%=============================================================================
\clearpage
\section{Primitive commands}
\label{sec:primitive-commands}
%=============================================================================

\begin{flalign*}
\pr{command}\ww \pp & \ww \pn{assignment}\ww &\\
\oo & \ww \pn{input}\ww \\
\oo & \ww \pn{output}\ww \\
\oo & \ww \pn{connect}\ww \\
\oo & \ww \pn{skip}\ww \\
\oo & \ww \pn{stop}
\end{flalign*}

A program is built from commands and the simplest of these are \emph{primitive}
commands, which are \emph{assignment}, \emph{connect}, \emph{input},
\emph{output}, \emph{skip} and \emph{stop}.


\subsection{Assignment}

\begin{flalign*}
&\pr{assignment} \pp \pn{variable}\ww\sm{$\colon$=}\ww\pn{expression}&
\end{flalign*}

% Semantics
An \emph{assignment command} evaluates the right hand side expression and
changes the value of the variable to the result. 


\subsection{Input and output}

\begin{flalign*}
\pr{input} & \pp \pn{chanend}\ww\sm{?}\ww\pn{variable}&\\
\pr{output} & \pp \pn{chanend}\ww\sm{!}\ww\pn{expression}&
\end{flalign*}

% Channels
Values are passed between processes using \emph{bidirectional communication
channels}. These consist of two \emph{channel ends} that are associated with
either process communicating on a channel.
% Synchronised communication
Communication is \emph{synchronised}, which means that an outputting process
must wait for the inputting process to be ready before any data is sent. This
prevents any data from being lost due to programming errors and precludes the
use of buffering.

% Semantics
An \emph{input command} receives a value from a channel and assigns it to a
variable and an \emph{output command} evaluates an expression and sends the
result on a channel.
%
Matching pairs of input and output are a distributed form of assignment.


\subsection{Connect}

\begin{flalign*}
\pr{connect} \pp & \ww\sm{connect}\ww\pn{chanend}\ww\sm{to}\ww\pn{chanend}&\\
\pr{chanend} \pp & \ww\pn{element}&
\end{flalign*}

% Channels
A \emph{connection} between two channel ends is established with a matching
pair of \emph{connect} commands.
% Semantics
Each connect command connects a \emph{local} channel end to the corresponding
\emph{remote} channel end. The commands terminate when the connection has been
established.
%
The channel end itself is an element that specifies a name or field
(see~\Sect{expressions-elements}).


% Skip and stop
\subsection{Skip and stop}

\begin{flalign*}
\pr{skip}\ww \pp & \ww \sm{skip}\ww &\\
\pr{stop}\ww \pp & \ww \sm{stop}\ww
\end{flalign*}

The command \ttw{skip} does nothing and terminates, the command
\ttw{stop} never terminates, causing the process to execute no more
actions. These commands are used primarily to explain the behaviour of other
constructs in the language.


% ============================================================================
\clearpage
\section{Structured commands}
\label{sec:structured-commands}
% ============================================================================

\begin{flalign*}
\pr{command}\ww \pp & \ww \pn{alternation}\ww &\\
\oo & \ww \pn{conditional}\ww \\
\oo & \ww \pn{loop}
\end{flalign*}

The structured commands \emph{conditional}, \emph{alternation} and
\emph{loop} are control constructs that are used to combine commands.


\subsection{Alternation}
\label{sec:alternation}

\begin{flalign*}
\pr{alternation}\ww \pp & \ww \sm{alt}\ww \sm{\{}\ww \{_{0}\ww\sm{|}\ww \pn{alternative}\ww \}\ww \sm{\}}\ww &\\
\pr{alternative}\ww \pp & \ww \pn{guarded-alternative}\ww \\
\oo & \ww \pn{alternation}\ww \\
\pr{guarded-alternative}\ww \pp & \ww \pn{guard}\ww \sm{:}\ww \pn{command}\ww &\\
\pr{guard}\ww \pp & \ww \pn{input}\ww \\
\oo & \ww \pn{expression}\ww \sm{\&}\ww \pn{input}\ww \\
\oo & \ww \pn{expression}\ww \sm{\&}\ww \pn{skip}
\end{flalign*}

% Semantics: alternative
An \emph{alternation command} is used to deal with non-determinism by allowing
a process to wait for inputs on a number of different channels. 
%
It consists of a list of \emph{guarded command alternatives}.  A guard consists
of an expression and an input on a channel or \ttw{skip} in the place of
the input if none is required.
%
When the value of the expression is \ttw{true}, the guard behaves like the input or
\ttw{skip}, otherwise it behaves like \ttw{stop} and does not proceed.
Conceptually, the guard expression \emph{enables} or \emph{disables} an
alternative.

% Semantics: alternation
When an alternation command is performed, it behaves like one of the
alternatives that can proceed. Without any alternatives, it behaves like
\ttw{stop}.
% Nested
An alternative that is itself an alternation is ready when one of its
alternatives is ready.


\subsection{Conditional}

\begin{flalign*}
\pr{conditional}\ww \pp & \ww \sm{if}\ww \sm{\{}\ww \{_{0}\ww\sm{|}\ww \pn{choice}\ww \sm{\}}\ww \\
\oo & \ww \sm{if}\ww \pn{expression}\ww \sm{then}\ww \pn{command}\ww \sm{else}\ww \pn{command}\ww &\\
\pr{choice}\ww \pp & \ww \pn{guarded-choice}\ww \\
\oo & \ww \pn{conditional}\ww \\
\oo & \ww \pn{specification}\ww \sm{:}\ww \pn{choice}\ww &\\
\pr{guarded-choice}\ww \pp & \ww \pn{expression}\ww \sm{:}\ww \pn{command}
\end{flalign*}

% Semantics
A \emph{conditional command} consists of a list of expressions enumerating a
set of choices, with a command associated with of each.  When a conditional
command is performed, each choice is tested in sequence and the command behaves
like the first choice that evaluates \ttw{true}. Without any choices, it behaves like
\ttw{skip}.

% If-then-else
An alternative \emph{if-then-else} form of the conditional command combines two
choices, one of which is performed when the value of a condition expression is
\ttw{true}, and the other when it is false.
% Semantics
Let $e$ be an expression and $C$ and $D$ be commands, then the alternative
conditional form is defined as 
\code{\enclose{\w{if} $e$ \w{then} $C$ \w{else} $D$} $=$ 
  \enclose{\w{if} \{ $e$: $C$ | \w{\textasciitilde}$e$: $D$ \}}}


\subsection{Loop}

\begin{flalign*}
\pr{loop}\ww \pp & \ww \sm{while}\ww\pn{expression}\ww\sm{do}\ww\pn{command}&
\end{flalign*}

% Semantics
A \emph{looping command} repetitively evaluates a condition expression and
executes a command if the value of the expression is \ttw{true}. When it is
\ttw{false}, it terminates. 


%% Case
%\subsubsection{Case}
%
%\begin{flalign*}
%\pr{case}\ww \pp & \ww \sm{case}\ww \sm{(}\ww \pn{expression}\ww \sm{)}\ww 
%    \sm{\{}\ww \{_{0} \sm{|}\ww \pn{selection}\ww \}\ww \sm{\}}\ww &\\
%\pr{selection}\ww \pp & \ww \pn{expression}\ww \sm{:}\ww \pn{command}\ww \\
%\oo & \ww \sm{else}\ww \pn{command}\ww &
%\end{flalign*}
%
%A \emph{case command} consists of a list of \emph{selections}, each with an
%associated value. When a case command is performed, the value of the
%\emph{selector} is evaluated, and if this is equal to one of the selection
%values, then the case command behaves like the command in the selection.
%Otherwise, it behaves like the command in the $\sm{else}$ selection, or like
%$\sm{stop}$ if there is no $\sm{else}$ selection. Without any selections, it
%behaves like $\sm{skip}$.



%=============================================================================
\clearpage
\section{Types, names and scope}
\label{sec:types-names-scope}
%=============================================================================


\subsection{Types}

\subsubsection{Primitive types}

\begin{flalign*}
\pr{type}\ww \pp & \ww \pn{primitive-type}\ww &\\
\pr{primitive-type}\ww \pp & \ww \sm{var}
\end{flalign*}

% Semantics
There is a single primitive \emph{variable} type in \sire. 
%
It holds a value that can be changed by input or assignment and it is specified
by the keyword \ttw{var}. 
% Interpretation
Variables are interpreted as signed integers and if no value has been assigned
to it then the value is arbitrary.


\subsubsection{Array types}

\begin{flalign*}
\pr{type}\ww \pp & \ww \pn{array-type}\ww &\\
\pr{array-type}\ww \pp & \ww \pn{primitive-type} \\
\oo & \ww \pn{array-type}\ww \sm{[}\ww \pn{expression}\ww \sm{]}
\end{flalign*}

% Semantics
An \emph{array variable} contains a number of component variables.  
%
Let $T$ be a primitive type, then a 1-dimensional array of components of type
$T$ is specified by the type $T$[$e$] where $e$ is an expression specifying the
length of the dimension.
% Multi-dimensional arrays
Arrays with additional dimensions can be declared by specifying additional
lengths in the type.
%
Let $T$ be a primitive type, then $T$[$e_1$][$e_2$]$\cdots$[$e_d$] specifies a
$d$-dimensional array with $e_1\times e_2\times\cdots\times e_d$ components of
type $T$.

% Array components Element
The components of an array variable are accessed with integer-valued subscripts
that can appear as an element in an expression or on the left hand side of an
assignment or input.
% Example
The \emph{element} $u$[$i$] selects the $i^\text{th}$ component of a
1-dimensional array variable with the name $u$, and
%
the subscript $v$[$i_1$][$i_2$]$\cdots$[$i_d$] selects the $i_1^\text{th}$
component of the first dimension and the $i_2^\text{th}$ component of the
second dimension etc.\ of a $d$-dimensional array variable with the name $v$.


\subsection{Scope}

A name has a context in which it is valid and can be used. 
This is called the \emph{scope}. 
%
The scope of a name extends from the point at which it is declared to the end
of the \emph{block}. 
%
A block is enclosed by curly braces $\w{\{}\cdots\w{\}}$ or a choice or
alternative.
%
The scope of a name also extends to any nested blocks.


\subsection{Declarations}

\begin{flalign*}
\pr{declaration}\ww \pp & \ww \pn{type}\ww \{_{1} \sm{,}\ww \pn{name}\ww \}\ww &
\end{flalign*}

% Semantics
A \emph{declaration} $T$ $n$ declares a variable of type $T$ with the name
$n$. 
%
A single declaration can specify multiple names of a particular type. 
Let $S(x)$ be a scope in which the name $x$ is free, $T$ be a type and $N_1,
N_2, \cdots, N_n$ be names, then 
\code{\enclose{$T$ $N_1, N_2, \cdots, N_n$ : $S$} $=$ 
  \enclose{$T$ $N_1$ : $T$ $N_2$ : $\cdots$ : $T$ $N_n$ : $S$}}


\subsection{Abbreviations}

\begin{flalign*}
\pr{abbreviation}\ww \pp & \ww \pn{specifier}\ww \pn{name}\ww \sm{is}\ww \pn{element}\ww \\
\oo & \ww \sm{val}\ww \pn{name}\ww \sm{is}\ww \pn{expression}\ww &\\
\pr{specifier}\ww \pp & \ww \pn{primitive-type}\ww \\
\oo & \ww \pn{specifier}\ww \sm{[}\ww \sm{]}\ww \\
\oo & \ww \pn{specifier}\ww \sm{[}\ww \pn{expression}\ww \sm{]}\ww &\\
\pr{element}\ww \pp & \ww \pn{element}\ww \sm{[}\ww \pn{expression}\ww \sm{]}\ww \\
\oo & \ww \pn{field}\ww &\\
\oo & \ww \pn{name}\ww &
\end{flalign*}

An \emph{abbreviation} can be used to specify new names for variables and
values.


\subsubsection{Variable abbreviation}

% Semantics
A \emph{variable abbreviation} specifies the name for an \emph{element} that
can either be a name or subscripted component of an array. 
%
Let $T$ be a type, $n$ be a name, $e$ be an element and $S$ be a scope. Then a
variable abbreviation has the effect \code{\enclose{$T$ $n$ \w{is} $e$ :
$S(n)$} $=$ \enclose{$S(e)$}}
% Arrays
The length specifier for an array dimension can be omitted from an abbreviation
to define a name of an array of components of the specified type of any length
in that dimension.
%
An array abbreviation is said to be \emph{compatible} with an array variable if
the lengths of all dimensions are equal or the abbreviation does not specify
the length of a particular dimension.


\subsubsection{Value abbreviation}

% Semantics
A \emph{value abbreviation} specifies a name for an expression. 
%
Let $n$ be a name, $e$ be an expression and $S$ be a scope. Then a value
abbreviation has the effect 
\code{$\lb$\w{val} $n$ \w{is} $e$ : $S(n)\rb$ $=$ $\lb S(e)\rb$}
%
The value of the abbreviation must remain constant throughout its use. To
ensure this, the abbreviated expression must not contain any variables that are
updated by assignment or input in the scope of the abbreviation.


\subsection{Specifications}
\label{sec:specifications}

\begin{flalign*}
\pr{specification}\ww \pp & \ww \pn{declaration}\ww \\
\oo & \ww \pn{abbreviation}\ww &\\
\pr{command}\ww \pp & \ww \pn{specification}\ww \sm{:}\ww \pn{command}\\
\pr{alternative}\ww \pp & \ww \pn{specification}\ww \sm{:}\ww \pn{alternative}
\end{flalign*}

% Semantics
A \emph{specification} introduces a new name into a scope and all names
specified within a block must be unique, however, a block can introduce a name
that is already in scope.
%
In general, a free variable of a block is \emph{bound} to the most recent name
that is still in scope.

% Formally
If $S(x)$ and $S(y)$ are scopes with the same behaviour, but $S(x)$ contains
the free variable $x$ wherever $S(y)$ contains the free variable $y$ and
vice versa, and $N(x)$ and $N(y)$ are identical specifications except that
$N(x)$ specifies the name $x$ and $N(y)$ specifies the name $y$, then
\code{\enclose{$N(x)$ : $S(x)$} $=$ \enclose{$N(y)$ : $S(y)$}}
%
In other words, specifications can always be added to produce a block in which
the free variables are distinct from any scope it may be inserted into. 
%
This provides a basis for the semantics of substitution for process
definitions, server definitions and functions, which are all explained
in~\Sect{procedural-abstraction}.


\subsection{Rules}
\label{sec:abbreviation-rules}

\ben

\item \emph{Declarations}. A specification can only introduce declarations that
  have a primitive variable type. 

\item \emph{Valid abbreviations}. The specifier of the element specified in the
  abbreviation must be compatible with the type of the name.  Compatibility
  requires the primitive types to be the same and, for array types, each
  specified length must be also be equal in the corresponding dimension
  (unspecified lengths can match any dimension).

\item \emph{Abbreviation subscripts}. Any variables used in subscripts of an
  abbreviated variable cannot be changed in the scope of the abbreviation.

\item \emph{Value abbreviations}.  The value of an abbreviated variable may be
  changed but to ensure the abbreviation always refers to the same array
  component, then no variable in the subscript expression of that component can
  be changed by assignment or input.

\item \emph{Array abbreviations}. Components of an array must be identified by
  a single name within a given scope to prevent aliasing. Therefore, components
  of an array that has one or more abbreviated components may also be referred
  to by abbreviations.

\een



%=============================================================================
\clearpage
\section{Composition}
\label{sec:composition}
%=============================================================================

\begin{flalign*}
\pr{command}\ww \pp & \ww \sm{\{}\ww\pn{sequence}\ww\sm{\}} &\\
\oo & \ww \sm{\{}\ww\pn{parallel}\ww\sm{\}}
\end{flalign*}

\subsection{Sequence}

\begin{flalign*}
\pr{sequence}\ww \pp &\ww \{_{0}\ww \sm{;}\ww \pn{command}\ww \} &
\end{flalign*}

% Semantics
A set of commands are composed in \emph{sequence} with the `\ttt{;}' separator.
Execution starts with the first component command and each subsequent command
is executed if and when the preceding command terminates. The sequence
terminates when the last component command terminates.


\subsection{Parallel}

\begin{flalign*}
\pr{parallel}\ww \pp & \ww \{_{0}\ww\sm{\&}\ww \pn{parallel-component}\ww \}\ww
&\\
\pr{parallel-component}\ww \pp & \ww \pn{process}
\end{flalign*}

% Semantics
A \emph{parallel command} creates new \emph{processes} that execute in
parallel, where a process is another command.
%
A set of processes are composed in parallel with the `\ttt{\&}' separator.  It
causes the component processes to be executed simultaneously and it terminates
if and only if all of the component processes have terminated.


\subsubsection{Process interfaces}

\begin{flalign*}
\pr{process}\ww \pp & \ww \pn{interface}\ww \sm{:}\ww \pn{command}\ww \\
\oo & \ww \pn{command}\ww &\\
\pr{interface}\ww \pp & \ww \sm{interface}\ww \sm{(}\ww \{_{0}\ww\sm{,}\ww \pn{declaration}\ww \}\ww \sm{)}\ww &\\
\pr{type}\ww \pp & \ww \pn{chanend-type}\ww &\\
\pr{chanend-type}\ww \pp & \ww \sm{chanend}\ww &
\end{flalign*}

Processes executing in parallel can communicate via channels and a process can
specify a number of local channel ends to do this.  
%
The set of \emph{local} channel ends belonging to a process constitutes its
\emph{interface}, which is specified before the body of a process.

Let $N_1, N_2, \cdots, N_n$ be names, then
\code{\w{interface} (\w{chanend} $N_1$, $N_2$, $\cdots$, $N_n$)}
specifies an interface that declares $n$ channel ends in the scope of the
process.
%
No other types may be declared in a process interface.

\subsubsection{Process names}

\begin{flalign*}
\pr{parallel-component}\ww \pp & \ww \pn{process-label}\ww \pn{process}\ww&\\
\pr{process-label}\ww \pp & \ww \pn{name}\ww \sm{is} &
\end{flalign*}

% Naming
Channel connections are established between component processes of a parallel
command by \emph{naming} the processes. This is done by prefixing a process with a
\emph{label} of the form `$\pn{name}\ww\sm{is}$'. 
% Compound names
The name of a process acts as a prefix for a \emph{compound name} for
the channels in the interface, where each one can be selected as a \emph{field}.
%
Let $P$ be a process and $N_1, N_2, \cdots, N_n$ be names, then \code{p \w{is}
\w{interface} (\w{chanend} $N_1$, $N_2$, $\cdots$, $N_n$): $P$} is a process
with the name \ttt{p} that specifies an interface with $n$ components. Each
component of the interface can be selected with the fields \ttt{p}.$N_1$,
\ttt{p}.$N_2$, $\cdots$, \ttt{p}.$N_n$.

% Channel connections
A process name is visible to all of the component processes in the parallel
command and a process that is not named and does not engage in any channel
communication is said to be \emph{anonymous}.
%
A process can make a connection to another parallel process by selecting the
channel end name as a field using the process' name and using it as the target
of a connect statement.


\subsubsection{Process arrays}

Arrays of processes can be constructed provided they
have the same interface. Components of a process array are selected using
integer-valued subscripts.
%
Let $P_1, P_2, \dots, P_n$ be processes with the same interface, then \code{p
  \w{is} \{$P_1$, $P_2$, $\cdots$, $P_n$\}} declares a \emph{process array}
containing $n$ processes. The component processes are selected with the
subscripts \ttt{p[0]} to \ttt{p[$n-1$]}. 
%
Process arrays can also be nested to create multidimensional arrays.


\subsection{Rules}

\ben[resume]

% Disjointness of variables
\label{sec:parallel-disjointness}
%
\item \emph{Process disjointness}.  Processes in a parallel command must be
  \emph{disjoint} and only update disjoint sets of variables. This is to
  eliminate the possibility of unwanted race conditions, i.e.\ in the absence
  of alternation (which is explained in the next section), leaving alternation
  as the only means of introducing non-determinism.

  A variable or component of an array variable can be \emph{changed} by
  assignment or input, or \emph{read} if it occurs as an operand or in an
  expression.
  %
  Disjointness is enforced by permitting \emph{read-only access} to variables
  or components of array variables shared between components of a parallel
  command, and \emph{exclusive write access} when a variable or array variable
  component appears in at most one component of a parallel command.
  %
  With arrays, components of an array variable may only be changed in parallel
  if it can be determined at compile time that the array subscripts are used in
  a linear combination to select disjoint components of the array.

\item \emph{Point-to-point channels}. No channel end can be connected to by
  more than one process. This is to ensure that no channel can be used for
  input or output by more than one process.

\item \emph{Channel end subscripts}.  If the target of a connect command is a
  component of a process array, the name will contain a subscript and the
  subscript must only use constants or replicator indices. This is so that the
  process to which it belongs can be identified at compile time.

\item \emph{Nesting}. Channel connections can only be established between
  component processes of a parallel. Therefore, channel end names can only be
  used by child processes of the single parallel command and they cannot be
  passed to parallel commands nested in a process.

% Components of arrays must have the same interface
\item \emph{Named process arrays}. Components of a named array must all present
  the same interface.

% Only channel ends in interfaces
\item \emph{Process interfaces}. A process interface can only declare channel
  end components.

\een



%=============================================================================
\clearpage
\section{Servers}
\label{sec:servers}
%=============================================================================

% Overview
A \emph{server} is a special type of process that consists of special
declarations that define its behaviour. It executes in parallel with the
processes in its scope, is only active in response to \emph{calls} and
terminates when control flow leaves the block in which it is defined.
%
\emph{Client} processes of a server make calls that behave in the same way to a local procedure call.

% Abstraction
Servers provide a mechanism for abstraction by building program components that
provide a service or \emph{subroutine}, and they can be combined into larger
\emph{server structures} to employ parallelism or distribute storage.


\subsection{Specification}

\begin{flalign*}
\pr{server}\ww \pp & \ww \pn{interface}\ww \sm{:}\ww \pn{server-specification}\ww &\\
\pr{server-specification}\ww \pp & \ww \pn{declaration}\ww \\
\oo & \ww \sm{\{}\ww \{_{1}\ww\sm{:}\ww \pn{declaration}\ww \}\ww \sm{\}}
\end{flalign*}

% Specification
The specification of a server consists of an interface that defines means of
external interaction (consisting of calls and channel ends) and set of
declarations that define its behaviour. These are \emph{initialisation
commands} that are performed before it responds to anything, responses to each
communication or call and \emph{finalisation commands} that are performed when
it terminates.


\subsubsection{Interface}

\begin{flalign*}
\pr{declaration}\ww \pp & \ww \sm{call}\ww \{_{1}\ww\sm{,}\ww \pn{name}\ww \sm{(}\ww \{_{0}\ww\sm{,}\ww \pn{formal}\ww \}\ww \sm{)}\ww \}\ww &\\
\pr{abbreviation}\ww \pp & \ww \sm{call}\ww \pn{name}\ww \sm{is}\ww \pn{name}
\end{flalign*}

A call declaration
\code{\w{call} $N$ ($f_1,f_2,\cdots,f_n$)}
specifies $N$ as a name for a call with the formal parameters $f_1, f_2,
\cdots, f_n$.
%
A single call declaration can specify multiple calls. Let $F_k$ be a sequence
of formals $f_1, f_2, \cdots, f_n$ where $n$ is an integer defined for the
sequence. 
\code{\enclose{\w{call} $N_1$ ($F_1$), $N_2$, ($F_2$), $\cdots$, $N_n$ ($F_n$)}
  $=$\\
  \lindent\enclose{\w{call} $N_1$ ($F_1$), \w{call} $N_2$, ($F_2$), 
    $\cdots$, \w{call} $N_n$ ($F_n$)}}


\subsubsection{Alternation and call guards}

\begin{flalign*}
\pr{declaration}\ww \pp & \ww \pn{alternation}\ww &\\
\pr{guard}\ww \pp & \ww \sm{accept}\ww \pn{name}\ww \sm{(}\ww \{_{0}\ww\sm{,}\ww \pn{formal}\ww \}\ww \sm{)}\ww \\
\oo & \ww \pn{expression}\ww \sm{\&}\ww \sm{accept}\ww \pn{name}\ww \sm{(}\ww \{_{0}\ww\sm{,}\ww \pn{formal}\ww \}\ww \sm{)}
\end{flalign*}

% Alternation
The main behaviour of a server is defined by an \emph{alternation declaration}
that specifies an alternation command (see~\Sect{alternation} for the
definition of alternation). This, as well as containing input alternatives, can
contain \emph{call} alternatives.

% Alternatives
An \emph{alternative} dealing with a call specifies the formal parameters that
must match the list of formals in the specification in the interface of the
corresponding call exactly, including the names, and a command to be executed
when the call is invoked by a client.
% Guarded calls
A call alternative may be guarded with an expression in the same way as an
input can, so that the call is ready only when the guard is \ttw{true}.  The
effect is that the server will not respond to a call while the guard is false.


\subsubsection{Initialisation and finalisation}

\begin{flalign*}
\pr{declaration}\ww \pp & \ww \sm{initial}\ww \pn{command}\ww &\\
\oo & \ww \sm{final}\ww \pn{command}\ww
\end{flalign*}

% Initial
An \emph{initialisation declaration} specifies a command to be executed before the alternation.
% Final
A \emph{finalisation declaration} specifies a command to be executed after the
alternation, when the scope of the server has terminated.


\subsection{Declarations}

\begin{flalign*}
\pr{declaration}\ww \pp & \ww \pn{server-declaration}\ww &\\
\oo & \ww \pn{hiding-declaration}\ww &\\
\oo & \ww \pn{simultaneous-declaration} 
\end{flalign*}

\subsubsection{Server declarations}

\begin{flalign*}
\pr{server-declaration}\ww \pp & \ww \pn{name}\ww \sm{is}\ww \pn{server}\ww &\\
\pr{server}\ww \pp & \ww \pn{server-array}\ww &\\
\pr{server-array}\ww \pp & \ww \sm{[}\ww \{_{1}\ww\sm{,}\ww \pn{server}\ww \}\ww \sm{]}
\end{flalign*}

% Single
A \emph{server declaration} \code{n \w{is} $S$} declares \ttt{n} as a name for
the server $S$ and \ttt{n} acts as a prefix for the compound names of the call
and channel names specified in the interface of the server.

% Array: different types, same interface
Arrays of servers can be constructed provided they have the same interface. 
%
Let $S_1, S_2, \dots, S_n$ be servers with the same interface, then \code{s
  \w{is} [$S_1$, $S_2$, $\cdots$, $S_n$]} declares a server array containing
$n$ servers that are selected with the subscripts \ttt{s[0]} to \ttt{s[$n-1$]}. 
%
Server arrays can also be nested to create multidimensional arrays.


\subsubsection{Hiding declarations}

\begin{flalign*}
\pr{hiding-declaration}\ww \pp & \ww \sm{from}\ww \sm{\{}\ww \{_{1}\ww\sm{:}\ww
\pn{declaration}\ww \}\ww \sm{\}}\ww \sm{interface}\ww \pn{name} &
\end{flalign*}

% Semantics
A \emph{hiding declaration} encloses a set of declarations and specifies one
name declared by these to be visible to the scope of the declaration.
%
This is used to create minimal interfaces with collections of servers to
perform abstraction.


\subsubsection{Simultaneous declarations}

\begin{flalign*}
\pr{simultaneous-declaration}\ww \pp & \ww \{_{0}\ww\sm{\&}\ww
\pn{declaration}\ww \} &
\end{flalign*}

% Semantics
Declarations separated by the symbol `\ttt{\&}' occur \emph{simultaneously} and
have the same scope. They are therefore visible to one another.
%
Simultaneous declarations are used to introduce mutually referential
declarations, such as servers that communicate with each other.


\subsection{Calls}

\begin{flalign*}
\pr{command}\ww \pp & \ww \pn{element}\ww \sm{(}\ww \{_{0} \sm{,}\ww
  \pn{actual}\ww \}\ww \sm{)}\ww &
\end{flalign*}

A \emph{server call} is a command that specifies the name of the server, name
of the call and a list of actual parameters.
% Semantics
The behaviour of a server call is defined in the following way. 
Let $f_1, f_2, \dots, f_n$ be formals,
$a_1, a_2, \dots, a_n$ be actuals, 
and $C$ be a command; then
\code{s \w{is} 
  \w{interface} (\w{call} c($f_1$, $f_2$, $\cdots$, $f_n$)):\\
$\ttt{  }$\w{alt} \{ \w{accept} c($f_1$, $f_2$, $\cdots$, $f_n$) \(C\) \}:\\
s.c ($a_1$, $a_2$, $\cdots$, $a_n$)}
%
has the effect 
\code{$f_1$ \w{is} $a_1$ : $f_2$ \w{is} $a_2$ : $\dots$ : $f_n$ \w{is} $a_n$ :
  $C$}


\subsection{Rules}

\ben[resume]

% Server channels
\item \emph{Server channels}. A server can use channels for communication and
  these are subject to the same rules as processes, except that connections can
  only be established with other servers with the same scope. 
  %
  Since channel inputs can only appear in the alternatives in an alternation
  declaration, output commands can only appear either in the command associated
  with an alternative or in the initialisation or finalisation commands.

\item \emph{Server calls}. A server must provide call guards for each of the
  calls specified in its interface.

\item \emph{Server interfaces}. A server interface can only contain channel
  end and call specifier components.

\item \emph{Server disjointness}. Servers are subject to the same variable- and
  channel-disjointness rules as processes (see~\Sect{parallel-disjointness}). 
  %
  In particular, a server can change a variable or component of an array by
  assignment or input if no other server or process in its scope can read or
  change that variable, and, a server can read the value from a variable or
  component of an array only if no other server or process changes the
  variable.

\item \emph{Named server arrays}. Components of a server array must all present
  the same interface.

\een



%=============================================================================
\clearpage
\section{Replication}
\label{sec:replication}
%=============================================================================

\begin{flalign*}
\pr{command}\ww \pp & \ww \sm{seq}\ww \pn{replicator}\ww\pn{command}\ww &\\
\pr{process}\ww \pp & \ww \sm{par}\ww \pn{replicator}\ww\pn{process}\ww &\\
\pr{conditional}\ww \pp & \ww \sm{if}\ww \pn{replicator}\ww \pn{choice}\ww &\\
\pr{alternation}\ww \pp & \ww \sm{alt}\ww \pn{replicator}\ww \pn{alternative}\ww &\\
\pr{server-declaration}\ww \pp & \ww \pn{name}\ww \sm{is}\ww \pn{replicator}\ww \pn{server}\ww \\
\oo & \ww \pn{name}\ww \sm{is}\ww \sm{[}\ww \pn{expression}\ww \sm{]}\ww \pn{server}\ww &\\
\pr{replicator}\ww \pp & \ww \sm{[}\ww \{_{1}\ww\sm{,}\ww \pn{index-range}\ww \}\ww \sm{]}\ww &\\
\pr{index-range}\ww \pp & \ww \pn{name}\ww \sm{=}\ww \pn{expression}\ww \sm{for}\ww \pn{expression}\ww \\
\oo & \ww \pn{name}\ww \sm{=}\ww \pn{expression}\ww \sm{for}\ww \pn{expression}\ww \sm{step}\ww \pn{expression}\ww &
\end{flalign*}

% Description
Replicators can be used to create a number of similar commands, components of a
construct, processes or servers.
%
A replicator consists of one or more \emph{index ranges}.  An index range
declares the name of an \emph{index} variable, a \emph{base} expression, a
\emph{count} expression and, optionally, a \emph{step} expression.  The base,
count and step define the range of values that an index variable takes.  This
name of the index variable is available to the replicated component and each
replicated instance takes a unique value of the index. 


\subsection{Constructs}

% Semantics
Let $X$ be one of \ttw{if}, \ttw{alt}, \ttw{seq} or \ttw{par};
%
$Y(i)$ be a choice, alternative, command or process corresponding to $X$ in
which $i$ is free;
%
$\oplus$ be one of the separators `\ttt{|}', `\ttt{;}' or `\ttt{\&}'
corresponding to $X$;
%
and $b$, $c$ and $s$ be integer values.
%
Then, the behaviour of a \emph{replicated conditional, alternative, sequence or
  parallel} with a single index range is defined by the following. If $c\geq0$:
%
  \code{\enclose{$X$ [$i$=$b$ \w{for} $c$ \w{step} $s$] $Y(i)$} $=$ 
    \enclose{$X$ \{ $Y(b)$ $\oplus$ $Y(b+s)$ $\oplus\cdots\oplus$
  $Y(b+(c-1)s)$ \}}}

\noindent 
If $c=0$:
\code{\enclose{\w{if} [$i$=$b$ \w{for} 0 \w{step} $s$] $Y(i)$} $=$ 
  \enclose{\w{stop}}}
\code{\enclose{\w{alt} [$i$=$b$ \w{for} 0 \w{step} $s$] $Y(i)$} $=$ 
  \enclose{\w{stop}}}
\code{\enclose{\w{seq} [$i$=$b$ \w{for} 0 \w{step} $s$] $Y(i)$} $=$ 
  \enclose{\w{skip}}}
\code{\enclose{\w{par} [$i$=$b$ \w{for} 0 \w{step} $s$] $Y(i)$} $=$ 
  \enclose{\w{skip}}}

\noindent
If $c<0$:
\code{\enclose{$X$ [$i$=$b$ \w{for} $c$ \w{step} $s$] $Y(i)$} $=$ 
  \enclose{\w{stop}}}

\noindent
For an index range without a step expression
\code{\enclose{$X$ [$i$=$b$ \w{for} $c$] $Y(i)$} $=$ 
  \enclose{$X$ [$i$=$b$ \w{for} $c$ \w{step} 1] $Y(i)$}}

% Multiple indices
\noindent
Let $I_k$ be a index range \ttt{$i_k=b_k$ \w{for} $c_k$}, where $i_k$ is a name
and $b_k$ and $c_k$ are integer values defined for the index range.
%
Let $b_i$ and $c_i$ for $1\leq i \leq d$ be integer values, then
the behaviour of \emph{replicators} with $d$ index ranges is defined by
\code{\enclose{$X$ [$I_1, I_2, \cdots, I_d$] $Y(i_1,i_2,\cdots,i_d)$} $=$\\
  \lindent\enclose{$X$ [$I_1$] \{ $X$ [$I_2$] \{ $\cdots$ \{ $X$ [$I_d$]
$Y(i_1,i_2,\cdots,i_d)$ \}\}\}}}


\subsection{Servers}

% Servers
Servers can also be replicated.
Let $S(i)$ be a server specification in which $i$ is free; $n$ be a name; and
$b$, $c$ and $s$ be integer values. Then, the behaviour of a \emph{replicated
server} with a single index range is defined by
\code{\enclose{[$i=b$ \w{for} $c$ \w{step} $s$] $S(i)$} $=$ 
  \enclose{[$S(b)$, $S(b+s)$, $\cdots$, $S(b+(c-1)s)$]}}
\code{\enclose{[$i=b$ \w{for} $c$] $S(i)$} $=$ 
  \enclose{[$i=b$ \w{for} $c$ \w{step} 1] $S(i)$}} 

% Multiple indices
\noindent
Let $I_k$ be a index range \ttt{$i_k=b_k$ \w{for} $c_k$}, where $i_k$ is a name
and $b_k$ and $c_k$ are integer values.
%
Then, the behaviour of replicators with $d$ index ranges is defined by
\code{\enclose{[$I_1$, $I_2$, $\cdots$, $I_d$] $N$} $=$ 
  \enclose{n \w{is} [$I_1$] [ [$I_2$] [ $\cdots$ [ [$I_d$] $N$ ]]]}}
%
When replicated instances of a server do not require their unique indices, then
a shorthand form of the replicator can be used. Let $e$ be an expression, then
%
\code{\enclose{[$e$] $N$} $=$ \enclose{[$i=0$ \w{for} $e$] $N$}}


\subsection{Rules}

\ben[resume]

\item \emph{Replicator indices}. The value of any replicator index cannot be
  changed by an assignment or input.

\een

%=============================================================================
\clearpage
\section{Expressions and elements}
\label{sec:expressions-elements}
%=============================================================================

% Expressions
An \emph{expression} produces a value and is composed of operands and
operators. They cannot cause any \emph{side-effects} by changing any external
state.
%
An \emph{operand} is either an element or literal or nested expression, which is
enclosed by parentheses.
%
There are no precedence rules, so nested expressions must be explicitly
bracketed.
%
An \emph{operator} takes either one or two operands and produces a value.
% Elements 
An \emph{element} is either a name, subscripted name, field, or function call.

A full specification of expressions and elements is given in~\Sect{operators}.


\subsection{Valof expressions}

\begin{flalign*}
\pr{valof}\ww \pp & \ww \sm{valof}\ww \pn{command}\ww \sm{result}\ww
  \pn{expression}\ww \\
  \oo & \ww \pn{specification}\ww\sm{:}\ww\pn{valof}\ww &\\
\pr{expression}\ww \pp & \ww \sm{(}\ww \pn{valof}\ww \sm{)}\ww &
\end{flalign*}

% Semantics
A \emph{valof} expression produces a value from a command.
%
A valof expression $\sm{valof}$ $C$ $\sm{result}$ $e$ is a block and the scope
of $C$ extends to the $\sm{result}$ expression $e$. 
%
It is evaluated by performing the command $C$ and then evaluating the
expression $e$ to produce the result.


\subsection{Rules}

\ben[resume]

% No side-effects
% NOTE: effects on locals dont count as side-effects.
\item \emph{Valof side-effects}.  A valof expression cannot cause
  side-effects by changing any external state. It cannot therefore make calls
  using a free name or assign to a free variable.

\een


%=============================================================================
\clearpage
\section{Procedural abstraction}
\label{sec:procedural-abstraction}
%=============================================================================

The details of a process, server or expression can be hidden in a component
that presents an interface to allow the behaviour to be considered abstractly.
These components are called \emph{procedures} and \emph{functions}.


\subsection{Procedures}
\label{sec:procedures}

\subsubsection{Definition}

\begin{flalign*}
\pr{definition}\ww \pp & \ww \pn{procedure}\ww \\
\pr{procedure}\ww \pp & \ww \sm{process}\ww \pn{name}\ww \sm{(}\ww \{_{0}\ww\sm{,}\ww \pn{formal}\ww \}\ww \sm{)}\ww \sm{is}\ww \pn{process}\ww \\
\oo & \ww \sm{server}\ww \pn{name}\ww \sm{(}\ww \{_{0}\ww\sm{,}\ww
  \pn{formal}\ww \}\ww \sm{)}\ww \sm{is}\ww \pn{server}\ww\\
\pr{formal}\ww \pp & \ww \pn{specifier}\ww \{_{1}\ww\sm{,}\ww \pn{name}\ww \}\ww \\
\oo & \ww \sm{val}\ww \{_{1}\ww\sm{,}\ww \pn{name}\ww \}\ww &
\end{flalign*}

A procedure specifies a name, a process or server and a set of
\emph{formal parameters}. Since there is no global scope in \sire\ the set of
formal parameters are the free elements of the process or server.

% Formals
Let $S$ be a specifier, then a formal parameter of the form $S$ $n$ specifies a
name $n$ with type given by $S$. 
%
A single formal can also specify multiple names.
% Name specifier
Let $S$ be a specifier and $N_1, N_2, \cdots, N_n$ be names, then \code{$S$
$N_1, N_2, \cdots, N_n$ $=$ $S$ $N_1$, $S$ $N_2$, $\cdots$, $S$ $N_n$} $S$ may
specify a variable or array variable of a primitive or channel end type, or a
compound channel end or array of compound channel ends.
% Value specifiers
For value specifications \code{\sm{val} $N_1, N_2, \cdots, N_n$ $=$ \w{val}
$N_1$, \w{val} $N_2$, $\cdots$, \w{val} $N_n$.}
% TODO: call formal parameters Definition
Let $X$ be one of \ttw{process} or \ttw{server}, $Y$ be a process or server
corresponding to $X$ and $f_1, f_2, \cdots, f_n$ be formal parameters. Then,
the definition \code{$X$ $N$ ($f_1,f_2,\cdots,f_n$) \w{is} $Y$} defines the
name $N$ to be a procedure that behaves like $Y$.
% Array types
If $f_i$ is an array type, it must specify the length of each dimension with an
expression that consists of only of constant values or variables specified by
other the other formals $f_1,f_2,\cdots,f_{i-1},f_{i+1},\cdots,f_n$.


\subsubsection{Channel end and server call parameters for procedures}

\begin{flalign*}
\pr{primitive-type}\ww \pp & \ww \pn{process-type}\ww \\
\oo & \ww \pn{server-type}\ww \\
\pr{process-type}\ww \pp & \ww \sm{process}\ww \pn{name}\ww \\
\oo & \ww \sm{process}\ww \pn{interface}\ww &\\
\pr{server-type}\ww \pp & \ww \sm{server}\ww \pn{name}\ww \\
\oo & \ww \sm{server}\ww \pn{interface}\ww &
\end{flalign*}

A procedure can specify parameters for channel ends and server calls.
%
These may be specified directly or with a compound name from a process or
server reference with the keywords \ttw{process} or \ttw{server}
respectively, its interface and a name.
%
The interface can either be specified \emph{explicitly} by listing the
components, or \emph{implicitly} by giving the name of a procedure that defines
a process or server with the interface.


\subsubsection{Passing procedures as parameters}

\begin{flalign*}
\pr{formal}\ww\pp & \ww \pn{call-type}\ww \{_{1}\ww\sm{,}\ww \pn{name}\ww \sm{(}\ww \{_{0}\ww\sm{,}\ww \pn{formal}\ww \}\ww \sm{)}\ww \}\ww \\
\pr{abbreviation}\ww \pp & \ww \pn{call-type}\ww \pn{name}\ww \sm{(}\ww \{_{0}\ww\sm{,}\ww \pn{formal}\ww \}\ww \sm{)}\ww \sm{is}\ww \pn{name}\ww &\\
\pr{call-type}\ww \pp & \ww \sm{process}\\
                      & \ww \sm{function}
\end{flalign*}

A procedure or server call can specify procedures as parameters.
%
This is so that different implementations of a procedure with the same formals
can be supplied to a procedure instance or server call.
%
With a procedure parameter to a server call, the supplied procedure is executed
remotely.

A formal parameter of the form
\code{\w{process} $N$ ($f_1$, $f_2$, $\cdots$, $f_n$\ttt)} 
specifies $N$ as the name for a procedure parameter.
%
A single procedure can specify multiple calls. Let $F_k$ be a sequence
of formals $f_1, f_2, \cdots, f_n$ where $n$ is an integer defined for the
sequence. 
\code{\enclose{\w{process} $N_1$ ($F_1$), $N_2$, ($F_2$), $\cdots$, $N_n$ ($F_n$)}
  $=$\\
  \lindent\enclose{\w{process} $N_1$ ($F_1$), \w{process} $N_2$, ($F_2$), 
    $\cdots$, \w{process} $N_n$ ($F_n$)}}


\subsection{Instances}
\label{sec:instances}

\begin{flalign*}
\pr{process}\ww \pp & \ww \pn{instance}\ww &\\
\pr{server}\ww \pp & \ww \pn{instance}\ww &\\
\pr{command}\ww \pp & \ww \pn{instance}\ww &\\
\pr{instance}\ww \pp & \ww \pn{name}\ww \sm{(}\ww \{_{0}\ww\sm{,}\ww
  \pn{actual}\ww \}\ww \sm{)}\ww &\\
\pr{actual}\ww \pp & \ww \pn{element}\ww \\
\oo & \ww \pn{expression}\ww &
\end{flalign*}

% Description
An \emph{instance} of a procedure is created by specifying its name and a list
of \emph{actual parameters}.
%
When an \emph{instance} of the procedure is created, each formal serves as the
left hand side of an abbreviation of an \emph{actual parameter}, $a$, that is
supplied to the instance. This provides a binding of each free variable to the
scope in which it is instantiated.
%
Each actual is an element or expression with a compatible type with the formal
parameter.

% Substitution
The scoping rules mean that an instance of a process type can be substituted
in-place for the process it names by inserting abbreviations of each formal
with the actual parameter.
%
This is defined in the following way.
Let $X$ be one of \ttw{process} or \ttw{server}, 
$Y$ be a process or server corresponding to $X$,
$f_1, f_2, \cdots, f_n$ be formals, $a_1,a_2,\cdots,a_n$ be actuals.
Then, if $X$ is a program in which no name is
specified more than once and it contains the definition
%
\code{$X$ $N$ ($f_1, f_2, \cdots, f_n$) \w{is} $Y$ }
%
then, within its scope
%
\code{$N$ ($a_1, a_2, \dots, a_n$) $=$ $f_1$ \w{is} $a_1$ : $f_2$ \w{is} $a_2$
  : $\dots$ : $f_n$ \w{is} $a_n$ : $P$}
%
provided each abbreviation is valid.
%
This allows procedures to be compiled either as a closed subroutine or by
substituting the body of the definition directly with the instance.


\subsection{Hiding definitions}

\begin{flalign*}
\pr{definition}\ww \pp & \ww\sm{server}\ww \pn{name}\ww \sm{(}\ww \{_{0} \sm{,}\ww
\pn{formal}\ww \}\ww \sm{)}\ww \sm{inherits}\ww \pn{hiding-declaration}\ww &
\end{flalign*}

% Description
A different way to define a server is to inherit an interface from a
server in a set of declarations. This can be used to combine a number of
servers into a single reusable module, exposing a single server declaration as
an interface.

% Semantics
Let $i_1, i_2, \cdots, i_m$ be specifiers for components of an interface,
then the definition
\code{\w{server} $N$($f_1$, $f_2$, $\cdots$, $f_n$) \w{inherits}\\
  $\ttt{  }$\w{from}\\
  $\ttt{  }$ $\vdots$\\
  $\ttt{  }$ s \w{is} \w{interface} ($i_1$, $i_2$, $\cdots$, $i_m$) \w{to} $D$\\
  $\ttt{  }$ $\vdots$\\
  $\ttt{  }$\w{interface} s}
%
defines $N$ as the name of a server type who inherits the interface from the
server with name \ttw{s} in the hiding declaration.
%
Let $X$ be a program in form where no name is specified more than once, then if
$X$ contains the above definition, in the scope of $N$
\code{\enclose{x \w{is} $N$($a_1,a_2,\cdots,a_n$) : $S$} $=$\\
$\ttt{  }$\enclose{$\ttt{  }$$f_1$ \w{is} $a_1$ : $f_2$ \w{is} $a_2$ : $\dots$ : $f_n$ \w{is} $a_n$ :\\
$\ttt{  }$$\ttt{  }$s \w{is} \w{interface}($i_1, i_2, \cdots, i_n$): $D$ : 
$\cdots$ : $S$}} 
where $S$ is the scope of the declaration.


\subsection{Functions}

\subsubsection{Definitions and instances}

\begin{flalign*}
\pr{definition}\ww \pp & \ww \sm{function}\ww \pn{name}\ww \sm{(}\ww \{_{0}
  \sm{,}\ww \pn{formal}\ww \}\ww \sm{)}\ww \sm{is}\ww \pn{valof}\ww &\\
\pr{expression}\ww \pp & \ww \pn{instance}
\end{flalign*}

A valof expression can be reused by defining a \emph{function}.
%
Each formal parameter of a function must be a value abbreviation and functions
cannot contain calls to processes.
%
The definition
\code{\w{function} $N$ ($f_1, f_2, \cdots, f_n$) \w{is} \w{valof} $C$
  \w{result} $e$}
defines $N$ as the name of a function with the valof expression \ttw{valof}
$C$ $\sm{result}$ $e$.

% Substitution
Let $X$ be a program in a form where no name is specified more than once, then
if $X$ contains the above function definition, in the scope of $N$
\code{\enclose{$N$ ($a_0, a_1, \cdots, a_n$)} $=$\\ 
\lindent\enclose{(\w{valof} $f_0$ \w{is} $a_0$ : $f_1$ \w{is} $a_1$
  : $\cdots$ : $f_n$ \w{is} $a_n$ : $C$ \w{result} $e$)}}
provided each abbreviation is valid.
%
This allows functions to be compiled either as a closed subroutine or by
substitution.


\subsubsection{Functions as parameters}

\begin{flalign*}
\pr{call-type}\ww \pp & \ww \sm{function}\ww &
\end{flalign*}

Functions can be passed as parameters in the same way as procedures.


\subsection{Rules}

\ben[resume]

\item \emph{Parameters}. The rules for procedures, hiding definitions and
  functions follow those for abbreviations (see~\Sect{abbreviation-rules}). The
  following two rules further define compatibility for interface, call and
  procedure types.

\item \emph{Matching interface parameters.} 
  %
  A compound name that references a single process or server can only be
  supplied as an actual parameter if it is compatible. This requires the
  interface to be the same as the one specified by the formal parameter. 

\item \emph{Matching call and procedure parameters}.  
  %
  A call or procedure name can only be supplied as an actual parameter if it is
  compatible. This requires (1) the type of the formal parameter are the same
  and (2) each of the types of the formals in the definition of the call or
  procedure are the same as the formals specified in the parameter (which are
  specified in braces).

\item \emph{Array parameters}. 
  %
  Each array parameter of a process or server type must specify its length by
  an expression that can contain only constant values or the names of other
  value parameters.

\item \emph{Recursion}.
  %
  Recursive procedures or functions are not permitted.

\een


%=============================================================================
\clearpage
\section{Program}
\label{sec:program}
%=============================================================================

\begin{flalign*}
\pr{program}\ww \pp & \ww \pn{program-specification}\ww \sm{:}\ww \pn{program}\ww \\
\oo & \ww \pn{sequence}\ww &\\
\pr{program-specification}\ww \pp & \ww \pn{specification}\ww \\
\oo & \ww \pn{definition}\ww &\\
\pr{definition}\ww \pp & \ww \pn{simultaneous-definition}\ww &\\
\pr{simultaneous-definition}\ww \pp & \ww \{_{0}\ww\sm{\&}\ww \pn{definition}\ww \}\ww &
\end{flalign*}

% Program
A \emph{program} is a single command sequence, or process, with a specification
that can contain definitions of process types, server types and functions.


\subsubsection{Simultaneous definitions}

Definitions separated by the symbol \ttw{\&} occur \emph{simultaneously} and
share the same scope; they are therefore visible to one another. Simultaneous
definitions are used to introduce mutually referential definitions, such as
processes or servers that communicate with each other.


\subsection{Rules}

\ben[resume]

\item \emph{No global scope}. A program specification can only contain
  definitions and abbreviations.

\een


%==============================================================================
\clearpage
\section{Syntax}
%==============================================================================

\subsection{Collected syntax}

The following sections list the \sire\ syntax, with each section related to a
particular feature.

\input{collected.tex}


\clearpage
\subsection{Ordered syntax}

The following lists each element of the \sire\ syntax in alphabetical order.

\input{ordered.tex}


\clearpage
\subsection{Operators}
\label{sec:operators}

A $\pn{binary-operator}$ $\oplus$ takes two operands $a$ and $b$ and produces a
value $a \oplus b$.
%
A binary \emph{arithmetic operators} takes signed integer operands and produces
a signed integer result.

\begin{center}
\begin{tabular}{ccc}
{\bf Symbol} & {\bf Meaning} & {\bf Definition}\\
\midrule
\ttt{+} & sum & $a+b$\\
\ttt{-} & difference & $a-b$\\ 
\ttt{*} & produce & $a\times b$\\
\ttt{/} & quotient & $a / b$\\
\ttt{rem} & remainder & $a \mod b$
\end{tabular}
\end{center}

\noindent 
A binary \emph{relational operator} takes two signed integer values
and produces the value \ttt{true} or \ttt{false}. 

\begin{center}
\begin{tabular}{ccc}
{\bf Symbol} & {\bf Meaning} & {\bf Definition}\\
\midrule
\ttt{=} & equality & $a=b$\\
\ttt{\textasciitilde=} & inequality & $a\neq b$\\
\ttt{<} & less than & $a < b$\\
\ttt{<=} & less than or equal & $a \leq b$\\
\ttt{>} & greater than & $a > b$\\
\ttt{>=} & greater than or equal & $a \geq b$
\end{tabular}
\end{center}

\noindent
A binary \emph{logical operator} takes two operands and produces a bitwise
result $b_i = a_i\oplus b_i$ for $0\leq i < n$.

\begin{center}
\begin{tabular}{ccc}
{\bf Symbol} & {\bf Meaning} & {\bf Definition}\\
\midrule
\ttt{or} & bitwise or & $b$ \ttt{or} $0$ $=$ $b$, $b$ \ttt{or} $1$ $=$ $1$\\
\ttt{and} & bitwise and & $b$ \ttt{and} $0$ $=$ $0$, $b$ \ttt{and} $1$ $=$ $b$\\
\ttt{xor} & bitwise exclusive or & $b$ \ttt{xor} $0$ $=$ $b$, 
  $b$ \ttt{xor} $1$ $=$ $1$, $1$ \ttt{xor} $1$ $=$ $0$, \\
\ttt{{<}<} & left bitwise shift & $b_i=b_{i+1}$, $b_0=0$\\
\ttt{>{>}} & right bitwise shift & $b_i=b_{i-1}$, $b_n-1=0$\\
\end{tabular}
\end{center}

\noindent
A $\pn{unary-operator}$ $\oplus$ takes a single operand $a$ and produces the value
$\oplus a$.

\begin{center}
\begin{tabular}{ccc}
{\bf Symbol} & {\bf Meaning} & {\bf Definition}\\
\midrule
\ttt{-} & negation & $0-a$ \\
\ttt{\textasciitilde} & bitwise not & 
    $\text{\ttt{\textasciitilde}}0=1$, $\text{\ttt{\textasciitilde}}1=0$\\
\end{tabular}
\end{center}


\subsection{Representation of values}
\label{appx:representation-values}

Signed values are represented as a two's complement bit-pattern. With a word
size of $n$, values in the range $-2^{n-1} \leq n < 2^{n-1}$ can be
represented.

The literal value \ttt{true} represents an all-1 bit pattern (the value $-1$) and the
literal value \ttt{false} represents an all-0 bit pattern (the value 0). The effect of this
is consistent with the logical not operation: \ttt{not} \ttt{false} $=$
\ttt{true} and \ttt{not} \ttt{true} = \ttt{false}.


\subsection{Character set}

A $\pn{character}$ can be any alphabetical character

\begin{flalign*}
  \hspace{5mm}&\sm{A B C D E F G H I J K L M N O P Q R S T U V W X Y Z}&\\
  \hspace{5mm}&\sm{a b c d e f g h i j k l m n o p q r s t u v w x y z}&
\end{flalign*}

\noindent
or any special character

\begin{flalign*}
  \hspace{5mm}&\sm{\_ + - = , . : ; ? \{ \} [ ] ( ) \# \& ! * @ | " '}&
\end{flalign*}

\noindent
A $\pn{digit}$ can be any numeric character

\begin{flalign*}
  \hspace{5mm}&\sm{0 1 2 3 4 5 6 7 8 9}&
\end{flalign*}

\noindent
A $\pn{hex-digit}$ can be any numeric character or

\begin{flalign*}
  \hspace{5mm}&\sm{A B C D E F a b c d e f}&
\end{flalign*}

\noindent
A $\pn{name}$ consists of a sequence of alphanumeric characters and underscores.


\subsection{Comments}

A comment is introduced with a `\ttt{\%}' symbol and continues until the end of
  the line.


%\begin{minipage}{\textwidth} % keep with next
\subsection{Keywords}

The following are keywords in \sire\ and cannot be used for names.

\input{keywords}


% Bibliography
%\phantomsection
\addcontentsline{toc}{section}{References}
\bibliographystyle{alpha}
\bibliography{refs}

\end{document}

